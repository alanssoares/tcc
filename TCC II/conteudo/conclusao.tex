\chapter{Considerações Finais}
\label{chap:conclusao}

A construção de agentes bípedes aut\^onomos não é uma tarefa trivial. Além 
do comportamento de baixo nível que é representado por movimentos como andar, levantar os braços, chutar uma bola, 
é necessário desenvolver o comportamento de alto nível, onde o agente deve analisar as informações que possui, e 
a partir delas escolher a melhor ação a ser executada.

Este trabalho apresentou o desenvolvimento de um modelo para realizar o mapeamento e planejamento 
de trajetórias em tempo real para nagevação de agentes humanóides em ambiente simulado 3D. O objetivo 
do trabalho foi ter um modelo que possa ser utilizado no desenvolvimento de aplicações que 
requerem uma resposta rápida e ao mesmo tempo segura para guiar um agente através de um ambiente din\^amico. Como exemplo, podemos citar o resgate de pessoas em acidentes 
de desabamento, um agente inteligente deve ser capaz de se locomover no meio de destroços de concreto e buscar os 
melhores caminhos para chegar até a vítima de forma segura. Outra aplicação possível é a contrução de 
robôs de serviços que precisam navegar em ambientes onde prestarão serviços diversos aos humanos de forma 
segura e eficaz.

O modelo foi desenvolvido utilizando a linguagem C++, que é utilizada pelo time BahiaRT. Além disso, outras ferramentas 
como o Scilab, foi utilizada no desenvolvimento do algoritmo descrito na seção \ref{sec:calcvel}. 

Uma das contribuições trabalho é o Trainer3D. Esta ferramenta poderá ser utilizada por outros times da liga de simulação 3D 
para realizar testes, coletar indicadores de desempenho e automatizar o processo de otimização.

A validação do modelo se deu através do desenvolvimento de uma aplicação de passe. Esta aplicação foi escolhida 
para aumentar o nível estratégico do time BahiaRT, proporcionando jogadas que aumentem as chances de gol. Além disso, 
uma das principais motivações foi o principal campeonato mundial de futebol de rob\^os, a RoboCup.

Como trabalho futuro, a resolução dos problemas identificados no desenvolvimento da aplicação de passe se viu necessária. 
Estas influenciam diretamente a execução do passe no futebol robótico. Dentre os trabalhos, o desenvolvimento de um chute 
din\^amico é um trabalho imprescindível, pois engloba o rápido posicionamento para realizar um chute para qualquer direção.