\begin{resumo}
%Escrever um texto que contemple todo o conteúdo do trabalho, com espaçamento
%1,5, justificado. Conforme as normas NBR 14724:2011 e NBR 6028:2003,da ABNT,
%o resumo é elemento obrigatório, constituído de parágrafo único; uma seqüência de
%frases concisas e objetivas e não de uma simples enumeração de tópicos, não
%ultrapassando 500 palavras, O resumo deve ressaltar o objetivo, o método, os
%resultados e as conclusões do documento. Deve-se usar o verbo na voz ativa e na
%terceira pessoa do singular. Devem ser seguido, logo abaixo, das palavras
%representativas do conteúdo do trabalho, isto é, palavras-chave e/ou descritores,
%que são palavras principais do texto, sendo de 3 a 5, separadas por ponto)

A necessidade da automatização e da construção de agentes robóticos capazes de realizar tarefas complexas, perigosas, 
delicadas, em ambientes din\^amicos, tornou-se um desafio aos investigadores. Alguns desses desafios incluem o mapeamento 
de ambientes din\^amicos, planejamento de trajetória e a predição de colisão. Este trabalho propõe um modelo 
para mapeamento e planejamento de trajetórias com predição de colisão para agentes robóticos em ambiente simulado. O modelo 
foi desenvolvido utilizando o time base do BahiaRT, desenvolvido pelo grupo de pesquisa ACSO/UNEB em parceria com o 
FCPortugal(FEUP-LIACC/Univ. Aveiro). A validação do projeto foi realizada utilizando uma aplicação de passe em futebol 
robótico integrado ao modelo proposto.

\textbf{Palavras-chaves}: Mapeamento. Planejamento de trajetórias. Predição. Simula\c{c}\~ao 3D.
\end{resumo}	
